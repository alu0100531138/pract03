\documentclass[a4paper,12pt]{article}
\usepackage[utf8]{inputenc}
\usepackage[spanish]{babel}
\begin{document}
\title{Título del artículo}
\author{Nombre y Apellido \\
        Tecnicas Experimentales~\footnote{Universidad de La Laguna}
        }
\date{\today}
\maketitle
\begin{abstract}
  En \LaTeX{}~\cite{Lam:86} es sencillo escribir expresiones
  matemáticas como $a=\sum_{i=1}~{10} {x_i}~{3}$
  y deben ser escritas entre dos simbolos \$.
  Los superindices se obtienen con el simbolo \~{},y
  los subindices con el simbolo \.
  Por ejemplo: $x~2 \ times y~{alpha + \beta}$
  También se pueden escribir fórmulas centradas:
  \[h~2=a~2 + b~2 \]
  \end{abstract}
  
  \section{Primera sección}
  \bigakip
  \begin{tabular}{|l|c|c|}
  \hline
       Nombre & Edad & Nota \\ \hline
       Pepe   &   24     10 \\ hline
       Juan   & 19&  8 \\ hline
       Luis  & 21&  9\\hline
       \end{tabular}
 Si simplemente se desea escribir texto normal en Latex,
 sin complicadas f\'ormulas matem|\'aticas o efectos especiales
 como cambios de fuentes,entonces simplemente tiene que escribir
 en espa\~nol normalmente.
 Si desea cambiar de párrafo ha de dejar una línea en blanco o bien 
 utilizar el comando \par
 No es necesario preocuparse de la sangría de los párrafo:
 todos  los párrafos se sangraran automaticamente con la excepción
 del primer párrafo de una seccion
 se ha de distinguir entre la comilla simple 'izquierda'
 y la comilla simple 'derecha' cuando se escribe en el ordenador
 En el caso de que se quieran utilizar comillas dobles se han de 
 escrbir dos caracteres 'comillas simple' seguidos,esto es,
 ``comillas dobles''.
 Tambien se ha de tener cuidado con los guiones: se utilizar un único 
 guion para la separación de sílabas,mientras que se utilizan
 tres guiones seguidos para producir un guion de los que se usan
 como signo de puntuación ---como en esta oración
 \begin{thebibliography}-{00}
   \bibitem{Lam:86}
   Lamport,Leslie.
   TLA in pictures.
   \emph{IEEE Transaction on Software Engineering}.
   21(9), 768-775.
   (1995)
   \end{thebibliography}
\end{document}
